%-------------------------
% Resume in Latex
% based on https://github.com/sb2nov/resume
% License : MIT
%------------------------

\documentclass[letterpaper,11pt]{article}

\usepackage{latexsym}
\usepackage[empty]{fullpage}
\usepackage{titlesec}
\usepackage{marvosym}
\usepackage[usenames,dvipsnames]{color}
\usepackage{verbatim}
\usepackage{enumitem}
% \usepackage[hidelinks]{hyperref}
\usepackage[colorlinks=true, citecolor=red, urlcolor=blue]{hyperref}
\usepackage{fancyhdr}
\usepackage[english]{babel}
\usepackage{tabularx}
\input{glyphtounicode}

\pagestyle{fancy}
\fancyhf{} % clear all header and footer fields
\fancyfoot{}
\renewcommand{\headrulewidth}{0pt}
\renewcommand{\footrulewidth}{0pt}

% Adjust margins
\addtolength{\oddsidemargin}{-0.5in}
\addtolength{\evensidemargin}{-0.5in}
\addtolength{\textwidth}{1in}
\addtolength{\topmargin}{-.5in}
\addtolength{\textheight}{1.0in}

\urlstyle{same}

\raggedbottom
\raggedright
\setlength{\tabcolsep}{0in}

% Sections formatting
\titleformat{\section}{
  \vspace{-5pt}\scshape\raggedright\large
}{}{0em}{}[\color{black}\titlerule \vspace{-5pt}]

% Ensure that generate pdf is machine readable/ATS parsable
\pdfgentounicode=1

%-------------------------
% Custom commands
\newcommand{\resumeItem}[2]{
  \item\small{
    \textbf{#1}{: #2 \vspace{-2pt}}
  }
}

\newcommand{\resumeSingle}[1]{
	\small{
		{#1 \vspace{0pt}}
	}
}

\newcommand{\resumeSingleItem}[1]{
	\item\small{
		{#1 \vspace{0pt}}
	}
}

\newcommand{\resumePubItem}[2]{
  \item\small{
    \textbf{#1}{ #2 \vspace{-2pt}}
  }
}

% Just in case someone needs a heading that does not need to be in a list
\newcommand{\resumeHeading}[4]{
    \begin{tabular*}{0.99\textwidth}[t]{l@{\extracolsep{\fill}}r}
      \textbf{#1} & #2 \\
      \textit{\small#3} & \textit{\small #4} \\
    \end{tabular*}\vspace{-5pt}
}

\newcommand{\resumeSubheading}[4]{
  \vspace{-1pt}\item
    \begin{tabular*}{0.97\textwidth}[t]{l@{\extracolsep{\fill}}r}
      \textbf{#1} & #2 \\
      \textit{\small#3} & \textit{\small #4} \\
    \end{tabular*}\vspace{+1pt}
}

\newcommand{\resumeSubSubheading}[2]{
    \begin{tabular*}{0.97\textwidth}{l@{\extracolsep{\fill}}r}
      \textit{\small#1} & \textit{\small #2} \\
    \end{tabular*}\vspace{-5pt}
}

\newcommand{\resumeSubItem}[2]{\resumeItem{#1}{#2}\vspace{-4pt}}

\renewcommand{\labelitemii}{$\circ$}

\newcommand{\resumeSubHeadingListStart}{\begin{itemize}[leftmargin=*]}
\newcommand{\resumeSubHeadingListEnd}{\end{itemize}}
\newcommand{\resumeItemListStart}{\begin{itemize}}
\newcommand{\resumeItemListEnd}{\end{itemize}\vspace{-5pt}}

%-------------------------------------------
%%%%%%  CV STARTS HERE  %%%%%%%%%%%%%%%%%%%%%%%%%%%%


\begin{document}

%----------HEADING-----------------
\begin{tabular*}{\textwidth}{l@{\extracolsep{\fill}}r}
    \textbf{\Large Shamil Iakubov} | Email : \href{mailto:yakubov.sha@gmail.com}{yakubov.sha@gmail.com} | Telegram : @shmiak \\
    A Data Scientist with extensive experience in research, optimization, and numerical modeling.\\
\end{tabular*}


%-----------EXPERIENCE-----------------
\section{Experience}
  \resumeSubHeadingListStart
	\resumeSubheading
	{shmyak.ai}{}{Data scientist}{2020-Now}
	\resumeItemListStart
	\resumeSingle{
		AI driven cryptocurrency trading on the spot market.
	}
		\resumeSingleItem{
			I have developed and deployed several spot market cryptocurrencies trading bots.
			There is a demonstration bot available: (\href{https://github.com/shmyak-ai/cryp-bot-demo/blob/main/CrypRLAgent.ipynb}{https://github.com/shmyak-ai/cryp-bot-demo/blob/main/CrypRLAgent.ipynb}).
			The different implementations are based on various reinforcement learning algorithms.
		}
		\resumeSingleItem{
			To explore the ability of reinforcement learning algorithms to be constructors, I have developed a custom Toy-Bridge builder environment in Unity. A short summary is here: \href{https://github.com/shmyak-ai/construction-unity-environments}{https://github.com/shmyak-ai/construction-unity-environments}.
		}
		\resumeSingleItem{
			Other projects include implementations of custom reinforcement learning training loops for fine graded learning control.
		}
	\resumeItemListEnd
		
    \resumeSubheading
      {Helmholtz-Zentrum}{Geesthacht, Germany}{Researcher}{2017 - 2020}
      \resumeItemListStart
      \resumeSingle{
        I have studied North Sea ecosystems using supervised and unsupervised machine learning: Published 4 peer-reviewed articles, 2 preprints, and 2 interactive jupyter notebook studies.
      }
          \resumeSingleItem{
          	Conducted several optimization studies. Example: \href{https://github.com/limash/Alkalinity_in_the_Wadden_Sea}{https://github.com/limash/Alkalinity\_in\_the\_Wadden\_Sea}.
            The research involved developing a model that predicts the ocean's ability to absorb carbon dioxide.
          }
          \resumeSingleItem{
          	Processed and analyzed geospatial data using Pandas, Numpy, Matplotlib, etc.        	Performed Exploratory data analysis in Jupyter notebooks.
          }
     \resumeItemListEnd
    
    \resumeSubheading
      {Institute of Oceanology}{Moscow, Russia}
      {Research engineer}{2014 - 2017}
      \resumeItemListStart
      \resumeSingle{
      	I was part of a team developing numerical models and performing signal processing.
      	Published 5 peer-reviewed articles.
      }
          \resumeSingleItem{
          	Conducted a signal processing study.
            It proposed a method to predict 'Freak waves' based on wave parameters.
      	    The study used unsupervised learning to categorize waves into groups and Fourier with Wavelet analysis to study properties and features of these groups.
          }
          \resumeSingleItem{
            Introduced CMake build system and refactored Fortran90 code to modern object-oriented Fortran in multiple numerical projects, introduced dynamic data i/o into several Fortran projects.
          }
          \resumeSingleItem{
          	Developed a diffusion-advection model of solutes and particulates transport in the ocean.
          }
          \resumeSingleItem{
          	Processed, analyzed and visualized waves data in MatLab using Signal Processing Toolbox and Wavelet Toolbox.
          }
      \resumeItemListEnd

  \resumeSubHeadingListEnd

%--------PROGRAMMING SKILLS------------
\section{Programming Skills}
\resumeSubHeadingListStart
\resumeSingleItem{
	Python: Tensorflow, Keras, PyTorch, Numpy, Pandas, Matplotlib, Ray, RLLib, tf-agents
}
\resumeSingleItem{
	Fortran, C\#, MatLab
}
\resumeSingleItem{
	Git, Unity, Unity-ML, Linux, bash, gdb, Jupyter Notebook, GCP, CMake, Docker
}
\resumeSubHeadingListEnd

%-----------ML-EXPERIENCE------------
\section{Machine Learning awards}
\resumeSubHeadingListStart

\resumeItem{Top 2\% solution in the Featured Simulation Kaggle Competition "Lux AI"}
{Ranked 19th out of 1186 teams as a solo participant: team shmyak \href{https://www.kaggle.com/c/lux-ai-2021/leaderboard}{https://www.kaggle.com/c/lux-ai-2021/leaderboard}.
	The solution is based on imitation and reinforcement learning.
}

\resumeSubHeadingListEnd

%-----------PUBLICATIONS--------------
\section{Publications}
  \resumeSubHeadingListStart
    \resumePubItem{}
      {\url{https://scholar.google.com/citations?hl=en&user=72diaAiPvoYC&view_op=list_works&sortby=pubdate}}
  \resumeSubHeadingListEnd

%-----------EDUCATION-----------------
\section{Education}
  \resumeSubHeadingListStart
    \resumeSubheading
      {Moscow State University}{Moscow, Russia}
      {Specialist, Oceanography}{2003 -- 2008}
  \resumeSubHeadingListEnd
  
  Online courses:
  {\href{https://coursera.org/share/738f07c3514e74b4b4a2ec4c7df53d11}{Deep Learning specialization},
   \href{https://coursera.org/share/a7e216966ca1e16fd6c58b7d58cb0806}{Machine Learning},
   \href{https://coursera.org/share/2a20ba63d42f5a2557db34e3e9837382}{Bayesian Statistics}.}

%-------------------------------------------
\end{document}
