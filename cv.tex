%-------------------------
% Resume in Latex
% based on https://github.com/sb2nov/resume
% License : MIT
%------------------------

\documentclass[letterpaper,11pt]{article}

\usepackage{latexsym}
\usepackage[empty]{fullpage}
\usepackage{titlesec}
\usepackage{marvosym}
\usepackage[usenames,dvipsnames]{color}
\usepackage{verbatim}
\usepackage{enumitem}
% \usepackage[hidelinks]{hyperref}
\usepackage[colorlinks=true, citecolor=red, urlcolor=blue]{hyperref}
\usepackage{fancyhdr}
\usepackage[english]{babel}
\usepackage{tabularx}
\input{glyphtounicode}

\pagestyle{fancy}
\fancyhf{} % clear all header and footer fields
\fancyfoot{}
\renewcommand{\headrulewidth}{0pt}
\renewcommand{\footrulewidth}{0pt}

% Adjust margins
\addtolength{\oddsidemargin}{-0.5in}
\addtolength{\evensidemargin}{-0.5in}
\addtolength{\textwidth}{1in}
\addtolength{\topmargin}{-.5in}
\addtolength{\textheight}{1.0in}

\urlstyle{same}

\raggedbottom
\raggedright
\setlength{\tabcolsep}{0in}

% Sections formatting
\titleformat{\section}{
  \vspace{-4pt}\scshape\raggedright\large
}{}{0em}{}[\color{black}\titlerule \vspace{-5pt}]

% Ensure that generate pdf is machine readable/ATS parsable
\pdfgentounicode=1

%-------------------------
% Custom commands
\newcommand{\resumeItem}[2]{
  \item\small{
    \textbf{#1}{: #2 \vspace{-2pt}}
  }
}

\newcommand{\resumePubItem}[2]{
  \item\small{
    \textbf{#1}{ #2 \vspace{-2pt}}
  }
}

% Just in case someone needs a heading that does not need to be in a list
\newcommand{\resumeHeading}[4]{
    \begin{tabular*}{0.99\textwidth}[t]{l@{\extracolsep{\fill}}r}
      \textbf{#1} & #2 \\
      \textit{\small#3} & \textit{\small #4} \\
    \end{tabular*}\vspace{-5pt}
}

\newcommand{\resumeSubheading}[4]{
  \vspace{-1pt}\item
    \begin{tabular*}{0.97\textwidth}[t]{l@{\extracolsep{\fill}}r}
      \textbf{#1} & #2 \\
      \textit{\small#3} & \textit{\small #4} \\
    \end{tabular*}\vspace{-5pt}
}

\newcommand{\resumeSubSubheading}[2]{
    \begin{tabular*}{0.97\textwidth}{l@{\extracolsep{\fill}}r}
      \textit{\small#1} & \textit{\small #2} \\
    \end{tabular*}\vspace{-5pt}
}

\newcommand{\resumeSubItem}[2]{\resumeItem{#1}{#2}\vspace{-4pt}}

\renewcommand{\labelitemii}{$\circ$}

\newcommand{\resumeSubHeadingListStart}{\begin{itemize}[leftmargin=*]}
\newcommand{\resumeSubHeadingListEnd}{\end{itemize}}
\newcommand{\resumeItemListStart}{\begin{itemize}}
\newcommand{\resumeItemListEnd}{\end{itemize}\vspace{-5pt}}

%-------------------------------------------
%%%%%%  CV STARTS HERE  %%%%%%%%%%%%%%%%%%%%%%%%%%%%


\begin{document}

%----------HEADING-----------------
\begin{tabular*}{\textwidth}{l@{\extracolsep{\fill}}r}
    \textbf{\Large Shamil Iakubov} : \href{https://www.linkedin.com/in/shamil-yakubov-93a08696/}{LinkedIn} & Email : \href{mailto:yakubov.sha@gmail.com}{yakubov.sha@gmail.com}\\
\end{tabular*}


%-----------ML-EXPERIENCE------------
\section{Machine Learning projects}
  \resumeSubHeadingListStart
    \resumeItem{Top 2\% solution in the Featured Simulation Kaggle Competition "Lux AI"}
        {Ranked 19th out of 1186 teams as a solo participant \href{https://www.kaggle.com/c/lux-ai-2021/leaderboard}{(team shmyak)}.
        The solution is based on imitation and distributed reinforcement learning.
        It consists of:
      
        1. \href{https://github.com/limash/lux-gym}{Environment.}
        An OpenAI gym wrapper for a Kaggle environment, which does preprocessing of raw data to provide ready observations to a trainer agent and contains several rule based agents.
        Several workers use the environment to collect experience to a data buffer.
      
        2. \href{https://github.com/limash/lux-ai}{Trainer.}
  	    It performs training of a function approximator, it includes different implementations of actor-critic and policy gradient algorithms, a custom training loop for imitation learning, and a custom buffer to store game experience.
        The buffer uses tfrecords files to prevent storing all experience trajectories in memory but consuming them efficiently from a storage device.
        It uses EfficientNetV2 squeeze-and-excitation layers as a function approximator.}
    
  	\resumeItem{Policy gradient based reinforcement learning for a \href{https://github.com/limash/gym-goose}{custom environment}}
  	    {An IMPALA style custom training loop
  	    \href{https://github.com/limash/goose_agent/blob/main/tf_reinforcement_agents/policy_gradient.py}{implementation} of an off-policy actor-critic algorithm with n-step update, policy gradients correction, entropy, and other improvements.
  	    It uses different convolutional neural nets as a function approximator and applies multi-attention for data preprocessing.}
    \resumeItem{DQN based reinforcement learning algorithms}
        {A custom training loops \href{https://github.com/limash/tf_reinforcement_testcases/blob/master/tf_reinforcement_testcases/deep_q_learning.py}{implementation} of several reinforcement algorithms (TensorFlow): DQN, FixedDQN, DoubleDQN, DoubleDuelingDQN, categorical DQN.
        It uses RAY to distribute calculations and DM Reverb as a data buffer to perform Prioritized Experience Replay.    
        It uses a \href{https://github.com/limash/tf_reinforcement_testcases/blob/b1d114437bde6f082657daf24eed66f1f79bac12/tf_reinforcement_testcases/models.py\#L64}{sparse MLP} as a function approximator.}
	\resumeItem{Modeling research}
		{An optimization \href{https://github.com/limash/Alkalinity_in_the_Wadden_Sea}{study}, which predicts carbon dioxide absorption capabilities of the ocean.
		It uses a custom function approximator and a non-linear least-squares minimization.}
	\resumeItem{Signal processing study}
		{It proposes the method to predict 'Freak waves' based on waves parameters.
		The \href{https://doi.org/10.3103/S1068373916110054}{study} uses cluster analysis to categorize waves to groups and then uses Fourier and Wavelet analysis to study properties and features of these groups.}

  \resumeSubHeadingListEnd


%-----------EXPERIENCE-----------------
\section{Numerical experience}
  \resumeSubHeadingListStart

    \resumeSubheading
      {Helmholtz-Zentrum in Geesthacht}{Geesthacht, Germany}{PhD student}{2017 - 2020}     
      \resumeItemListStart
        \resumeItem{Research}
          {Study of biogeochemical interactions between the ocean and the atmosphere in the North Sea region. Writing and publishing scientific papers. Processing and analyzing geospatial data using Pandas, Numpy, Matplotlib, etc.}
        \resumeItem{Development}
          {Building and optimization of ocean ecosystem and biogeochemical models.}
     \resumeItemListEnd
    
    \resumeSubheading
      {Institute of Oceanology}{Moscow, Russia}
      {Research engineer}{2014 - 2017}
      \resumeItemListStart
      	\resumeItem{Research}
      	  {Processing and analysis of ocean waves.
      	   Studying Arctic ecosystems and biogeochemical processes.}
        \resumeItem{Development}
          {Introducing the CMake build system to multiple Fortran projects.
           Migration from Fortran90 to modern object-oriented Fortran.
           Development of a diffusion-advection model of particle transport in the ocean.}
      \resumeItemListEnd

  \resumeSubHeadingListEnd

%
%--------PROGRAMMING SKILLS------------
\section{Programming Skills}
\resumeSubHeadingListStart
\item{
	\textbf{Languages}{: Python (Tensorflow, Keras, Numpy, Pandas), FORTRAN, LaTeX}
	% \hfill
	% \textbf{Technologies}{: AWS, Play, React, Kafka, GCE}
}
\resumeSubHeadingListEnd

%-----------PUBLICATIONS--------------
\section{Recent publications}
  \resumeSubHeadingListStart
    \resumePubItem{}
      {Yakubov, S.; Protsenko, E. Alkalinity Generation in the Coastal Area, the Case of the Wadden Sea. Preprints 2021, 2021020036 (\href{https://doi.org/10.20944/preprints202102.0036.v1}{doi:10.20944/preprints202102.0036.v1})}
%    \resumePubItem{}
%      {Yakushev, E.V.; Wallhead, P.; Renaud, P.E.; Ilinskaya, A.; Protsenko, E.; Yakubov, S.; Pakhomova, S.; Sweetman, A.K.; Dunlop, K.; Berezina, A.; Bellerby, R.G.J.; Dale, T. Understanding the Biogeochemical Impacts of Fish Farms Using a Benthic-Pelagic Model. Water 2020, 12, 2384. (\href{https://doi.org/10.3390/w12092384}{doi:10.3390/w12092384})}
%    \resumePubItem{}
%      {Yakubov, S.; Wallhead, P.; Protsenko, E.; Yakushev, E.; Pakhomova, S.; Brix, H. A 1-Dimensional Sympagic–Pelagic–Benthic Transport Model (SPBM): Coupled Simulation of Ice, Water Column, and Sediment Biogeochemistry, Suitable for Arctic Applications. Water 2019, 11, 1582. (\href{https://doi.org/10.3390/w11081582}{doi:10.3390/w11081582})}
%    \resumePubItem{}
%      {Pakhomova, S.; Yakushev E.; Protsenko E.; Rigaud S.; Cossa D.; Knoery J.; Couture R.; Radakovitch O.; Yakubov S.; Krzeminska D.; Newton A.
%      Modeling the Influence of Eutrophication and Redox Conditions on Mercury Cycling at the Sediment-Water Interface in the Berre Lagoon. Frontiers in Marine Science 2018, 5, 291.
%      (\href{https://doi.org/10.3389/fmars.2018.00291}{doi:10.3389/fmars.2018.00291})}
  \resumeSubHeadingListEnd


%-----------EDUCATION-----------------
\section{Education}
  \resumeSubHeadingListStart
    \resumeSubheading
      {Moscow State University}{Moscow, Russia}
      {Specialist, Oceanography}{2003 -- 2008}
  \resumeSubHeadingListEnd
  
  Online courses:
  {\href{https://coursera.org/share/738f07c3514e74b4b4a2ec4c7df53d11}{Deep Learning specialization},
   \href{https://coursera.org/share/a7e216966ca1e16fd6c58b7d58cb0806}{Machine Learning},
   \href{https://coursera.org/share/2a20ba63d42f5a2557db34e3e9837382}{Bayesian Statistics}.}



%-------------------------------------------
\end{document}
