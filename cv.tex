%-------------------------
% Resume in Latex
% based on https://github.com/sb2nov/resume
% License : MIT
%------------------------

\documentclass[letterpaper,11pt]{article}

\usepackage{latexsym}
\usepackage[empty]{fullpage}
\usepackage{titlesec}
\usepackage{marvosym}
\usepackage[usenames,dvipsnames]{color}
\usepackage{verbatim}
\usepackage{enumitem}
% \usepackage[hidelinks]{hyperref}
\usepackage[colorlinks=true, citecolor=red, urlcolor=blue]{hyperref}
\usepackage{fancyhdr}
\usepackage[english]{babel}
\usepackage{tabularx}
\input{glyphtounicode}

\pagestyle{fancy}
\fancyhf{} % clear all header and footer fields
\fancyfoot{}
\renewcommand{\headrulewidth}{0pt}
\renewcommand{\footrulewidth}{0pt}

% Adjust margins
\addtolength{\oddsidemargin}{-0.5in}
\addtolength{\evensidemargin}{-0.5in}
\addtolength{\textwidth}{1in}
\addtolength{\topmargin}{-.5in}
\addtolength{\textheight}{1.0in}

\urlstyle{same}

\raggedbottom
\raggedright
\setlength{\tabcolsep}{0in}

% Sections formatting
\titleformat{\section}{
  \vspace{-5pt}\scshape\raggedright\large
}{}{0em}{}[\color{black}\titlerule \vspace{-7pt}]

% Ensure that generate pdf is machine readable/ATS parsable
\pdfgentounicode=1

%-------------------------
% Custom commands
\newcommand{\resumeItem}[2]{
  \item\small{
    \textbf{#1}{: #2 \vspace{-2pt}}
  }
}

\newcommand{\resumeSingle}[1]{
	\small{
		{#1 \vspace{0pt}}
	}
}

\newcommand{\resumeSingleItem}[1]{
	\item\small{
		{#1 \vspace{0pt}}
	}
}

\newcommand{\resumePubItem}[2]{
  \item\small{
    \textbf{#1}{ #2 \vspace{-2pt}}
  }
}

% Just in case someone needs a heading that does not need to be in a list
\newcommand{\resumeHeading}[4]{
    \begin{tabular*}{0.99\textwidth}[t]{l@{\extracolsep{\fill}}r}
      \textbf{#1} & #2 \\
      \textit{\small#3} & \textit{\small #4} \\
    \end{tabular*}\vspace{-5pt}
}

\newcommand{\resumeSubheading}[4]{
  \vspace{-1pt}\item
    \begin{tabular*}{0.97\textwidth}[t]{l@{\extracolsep{\fill}}r}
      \textbf{#1} & #2 \\
      \textit{\small#3} & \textit{\small #4} \\
    \end{tabular*}\vspace{+1pt}
}

\newcommand{\resumeSubSubheading}[2]{
    \begin{tabular*}{0.97\textwidth}{l@{\extracolsep{\fill}}r}
      \textit{\small#1} & \textit{\small #2} \\
    \end{tabular*}\vspace{-5pt}
}

\newcommand{\resumeSubItem}[2]{\resumeItem{#1}{#2}\vspace{-4pt}}

\renewcommand{\labelitemii}{$\circ$}

\newcommand{\resumeSubHeadingListStart}{\begin{itemize}[leftmargin=*]}
\newcommand{\resumeSubHeadingListEnd}{\end{itemize}}
\newcommand{\resumeItemListStart}{\begin{itemize}}
\newcommand{\resumeItemListEnd}{\end{itemize}\vspace{-5pt}}

%-------------------------------------------
%%%%%%  CV STARTS HERE  %%%%%%%%%%%%%%%%%%%%%%%%%%%%


\begin{document}

%----------HEADING-----------------
\begin{tabular*}{\textwidth}{l@{\extracolsep{\fill}}r}
    \textbf{\Large Shamil Iakubov} 
    | Email : \href{mailto:yakubov.sha@gmail.com}{yakubov.sha@gmail.com} 
    | GitHub: \href{https://github.com/limash}{https://github.com/limash} \\
    | Telegram : @shmiak 
    | Google Scholar : \href{https://scholar.google.com/citations?user=72diaAiPvoYC&hl=en}{publications}
    | LinkedIn : \href{https://www.linkedin.com/in/shamil-iakubov-93a08696/}{profile} \\
    \\
    Oceanographer specializing in machine learning, AI-driven applications, numerical ocean modeling, and software \\
    development. Passionate about leveraging numerical modeling and machine learning to solve real-world challenges. \\
    Currently focused on applying machine learning advancements to improve marine ecosystem and biogeochemistry \\
    modeling.
\end{tabular*}


%-----------EXPERIENCE-----------------
\section{Experience}
  \resumeSubHeadingListStart

	\resumeSubheading
	{Engineer-oceanographer}{Oslo, Norway}{The Norwegian Institute for Water Research (NIVA)}{2022-now}
	\resumeItemListStart
	\resumeSingle{
		At NIVA, Shamil works on AI applications in oceanography (e.g. machine learning driven algal bloom prediction), 
        digital twins of the ocean (e.g. a data-driven ocean model implemented in the modern programming language Julia), 
        software development (user interface and logic for Ferrybox sensors: pH, pCO2, etc.), 
        and hydrophysical / biogeochemical ocean models development and application 
        (e.g. applying Regional Ocean Model System data assimilation to perform Observing System Simulation Experiments).
	}

	\resumeSingle{
		Skills: Julia (Programming Language), Python, Fortran, CUDA, 
        Functional Programming, Machine Learning, Biogeochemistry, Data Assimilation,
        High-performance computing - Sigma2 - FRAM.
	}
	\resumeItemListEnd

	\resumeSubheading
	{Data scientist}{}{Self-employed}{2020-2022}
	\resumeItemListStart
	\resumeSingle{
		AI bots programming; reinforcement learning environments creation; TF-records data pipelines; financial ML.
	}

	\resumeSingle{
		Skills: Reinforcement Learning, Python (TensorFlow, Jax, Ray), C\#, Unity, Unity-ML, Docker, GCP.
	}
	\resumeItemListEnd
		
    \resumeSubheading
    {Research Assistant, phd student}{Geesthacht, Germany}{Helmholtz-Zentrum}{2017 - 2020}
    \resumeItemListStart
    \resumeSingle{
        Studies of the marine ecosystem of the North Sea using numerical modeling, supervised and unsupervised machine learning.
    }

	\resumeSingle{
		Skills: Python, Fortran, Carbonate System modeling, LaTeX.
	}
    \resumeItemListEnd
    
    \resumeSubheading
    {Research engineer}{Moscow, Russia}{Institute of Oceanology}{2014 - 2017}
    \resumeItemListStart
    \resumeSingle{
        Development of transport and biogeochemical numerical models of the ocean; 
        signal processing and frequency analysis of ocean waves.
    }

	\resumeSingle{
		Skills: Fortran, MATLAB, Signal Processing, Linux, Vim, Git, GitHub.
	}
    \resumeItemListEnd

  \resumeSubHeadingListEnd

%-----------ML-EXPERIENCE------------
\section{Machine Learning awards}
\resumeSubHeadingListStart

\resumeItem{Top 2\% solution in the Featured Simulation Kaggle Competition "Lux AI"}
{Ranked 19th out of 1186 teams as a solo participant (Silver Medal): team shmyak \href{https://www.kaggle.com/c/lux-ai-2021/leaderboard}{https://www.kaggle.com/c/lux-ai-2021/leaderboard}.
	The solution is based on imitation and reinforcement learning.
}

\resumeSubHeadingListEnd

%-----------PUBLICATIONS--------------
\section{Publications}
  \resumeSubHeadingListStart
    \resumePubItem{Reinforcement learning cryptocurrency trading bot, proof of concept}{
        \href{https://medium.com/@yakubov.sha/reinforcement-learning-cryptocurrency-trading-bot-proof-of-concept-fda6b5c821a0}{Medium, 2022}}
    \resumePubItem{Alkalinity generation in the coastal area, the case of the Wadden Sea}{
        \href{https://www.preprints.org/manuscript/202102.0036/v1}{Preprints, 2021}}
    \resumePubItem{A 1-Dimensional Sympagic–Pelagic–Benthic Transport Model (SPBM): Coupled Simulation of Ice, Water Column, and Sediment Biogeochemistry, Suitable for Arctic Applications}{
        \href{https://doi.org/10.3390/w11081582}{MDPI, 2019}}
  \resumeSubHeadingListEnd

%-----------EDUCATION-----------------
\section{Education}
  \resumeSubHeadingListStart
    \resumeSubheading
      {Moscow State University}{Moscow, Russia}
      {Specialist, Oceanography}{2003 -- 2008}
  \resumeSubHeadingListEnd
  
  Online courses:
  {\href{https://coursera.org/share/738f07c3514e74b4b4a2ec4c7df53d11}{Deep Learning specialization},
   \href{https://coursera.org/share/a7e216966ca1e16fd6c58b7d58cb0806}{Machine Learning},
   \href{https://coursera.org/share/2a20ba63d42f5a2557db34e3e9837382}{Bayesian Statistics}.}

%-------------------------------------------
\end{document}
