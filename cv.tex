%-------------------------
% Resume in Latex
% based on https://github.com/sb2nov/resume
% License : MIT
%------------------------

\documentclass[letterpaper,11pt]{article}

\usepackage{latexsym}
\usepackage[empty]{fullpage}
\usepackage{titlesec}
\usepackage{marvosym}
\usepackage[usenames,dvipsnames]{color}
\usepackage{verbatim}
\usepackage{enumitem}
% \usepackage[hidelinks]{hyperref}
\usepackage[colorlinks=true, citecolor=red, urlcolor=blue]{hyperref}
\usepackage{fancyhdr}
\usepackage[english]{babel}
\usepackage{tabularx}
\input{glyphtounicode}

\pagestyle{fancy}
\fancyhf{} % clear all header and footer fields
\fancyfoot{}
\renewcommand{\headrulewidth}{0pt}
\renewcommand{\footrulewidth}{0pt}

% Adjust margins
\addtolength{\oddsidemargin}{-0.5in}
\addtolength{\evensidemargin}{-0.5in}
\addtolength{\textwidth}{1in}
\addtolength{\topmargin}{-.5in}
\addtolength{\textheight}{1.0in}

\urlstyle{same}

\raggedbottom
\raggedright
\setlength{\tabcolsep}{0in}

% Sections formatting
\titleformat{\section}{
  \vspace{-4pt}\scshape\raggedright\large
}{}{0em}{}[\color{black}\titlerule \vspace{-5pt}]

% Ensure that generate pdf is machine readable/ATS parsable
\pdfgentounicode=1

%-------------------------
% Custom commands
\newcommand{\resumeItem}[2]{
  \item\small{
    \textbf{#1}{: #2 \vspace{-2pt}}
  }
}

\newcommand{\resumePubItem}[2]{
  \item\small{
    \textbf{#1}{ #2 \vspace{-2pt}}
  }
}

% Just in case someone needs a heading that does not need to be in a list
\newcommand{\resumeHeading}[4]{
    \begin{tabular*}{0.99\textwidth}[t]{l@{\extracolsep{\fill}}r}
      \textbf{#1} & #2 \\
      \textit{\small#3} & \textit{\small #4} \\
    \end{tabular*}\vspace{-5pt}
}

\newcommand{\resumeSubheading}[4]{
  \vspace{-1pt}\item
    \begin{tabular*}{0.97\textwidth}[t]{l@{\extracolsep{\fill}}r}
      \textbf{#1} & #2 \\
      \textit{\small#3} & \textit{\small #4} \\
    \end{tabular*}\vspace{-5pt}
}

\newcommand{\resumeSubSubheading}[2]{
    \begin{tabular*}{0.97\textwidth}{l@{\extracolsep{\fill}}r}
      \textit{\small#1} & \textit{\small #2} \\
    \end{tabular*}\vspace{-5pt}
}

\newcommand{\resumeSubItem}[2]{\resumeItem{#1}{#2}\vspace{-4pt}}

\renewcommand{\labelitemii}{$\circ$}

\newcommand{\resumeSubHeadingListStart}{\begin{itemize}[leftmargin=*]}
\newcommand{\resumeSubHeadingListEnd}{\end{itemize}}
\newcommand{\resumeItemListStart}{\begin{itemize}}
\newcommand{\resumeItemListEnd}{\end{itemize}\vspace{-5pt}}

%-------------------------------------------
%%%%%%  CV STARTS HERE  %%%%%%%%%%%%%%%%%%%%%%%%%%%%


\begin{document}

%----------HEADING-----------------
\begin{tabular*}{\textwidth}{l@{\extracolsep{\fill}}r}
    \textbf{\Large Shamil Iakubov} : \href{https://github.com/limash}{https://github.com/limash} & Email : \href{mailto:yakubov.sha@gmail.com}{yakubov.sha@gmail.com}\\
\end{tabular*}


%-----------ML-EXPERIENCE------------
\section{Machine Learning experience}
  \resumeSubHeadingListStart
    \resumeSubItem{\href{https://github.com/limash/tf_reinforcement_agents}{TF2 reinforcement testcases} development}
      {Python3 TensorFlow2 reinforcement algorithms implementation: different versions of DQN, categorical DQN, off policy actor-critic algorithms with dueling networks, n-step update, off policy policy gradient correction and other improvements. 
      It uses \href{https://docs.ray.io/en/master/index.html}{RAY} to distribute calculations and \href{https://github.com/deepmind/reverb}{DM Reverb} as a data buffer.
      Some versions of it include sparse nets and residual convolutional nets.
      }
    \resumeSubItem{\href{https://github.com/limash/gym-goose}{Kaggle Geese competition Gym environment wrapper} development}
    {A \href{https://gym.openai.com/docs/}{Gym} wrapper around Kaggle Geese environment to sample experience efficiently with several reward systems and observation representations to use with the TF2 reinforcement testcases.}
    \resumeSubItem{Online courses}
    {\href{https://coursera.org/share/738f07c3514e74b4b4a2ec4c7df53d11}{Deep Learning specialization},
     \href{https://coursera.org/share/a7e216966ca1e16fd6c58b7d58cb0806}{Machine Learning},
     \href{https://coursera.org/share/2a20ba63d42f5a2557db34e3e9837382}{Bayesian Statistics}.}
  \resumeSubHeadingListEnd


%-----------EXPERIENCE-----------------
\section{Other experience}
  \resumeSubHeadingListStart

    \resumeSubheading
      {Helmholtz-Zentrum in Geesthacht}{Geesthacht, Germany}
      {PhD student}{2017 - 2020}
      \resumeItemListStart
        \resumeItem{Research}
          {\href{https://github.com/limash/Alkalinity_in_the_Wadden_Sea}{Study 1.}
          A modelling study of the Wadden Sea biogeochemistry features providing Jupyter notebooks with methods for easy reproducibility.
          \href{https://github.com/limash/Carbon_dioxide_absorption_drivers}{Study 2.} 
          A modelling study about controlling factors of the atmosphere - seawater carbon dioxide exchange in the area of the North Sea, in a Jupyter notebooks format.
           Both studies are computational heavy and based on biogeochemical models implemented in Python3 and Fortran2003.}
        \resumeItem{Biogeochemical and marine ecosystem models development}
          {Building and optimization of biogeochemical \href{https://github.com/BottomRedoxModel/brom_niva_module/tree/dev-sham}{models}.}
        \resumeItem{Data analysis}
          {\href{https://github.com/limash/helgoland_alkalinity}{Visualization and processing} of oceanographical data from the North Sea.}
      \resumeItemListEnd
      
    \resumeSubheading
      {Institute of Oceanology}{Moscow, Russia}
      {Junior Researcher}{2013 - 2017}
      \resumeItemListStart
        \resumeItem{Participation in development of BottomRedOxModel}
          {Responsobilities: 
          Add a computanionaly efficient pH calculation; 
          migrate from Fortran77 to Fortran2003;
          migrate from Visual Studio solutions to CMake;
          add Linux support.}
        \resumeItem{Sympagic-Pelagic-Benthic-Model development}
          {\href{https://github.com/BottomRedoxModel/SPBM}{A 1-dimensional biogeochemical tracers transport model.}
          The model solves numerically a system of 1-D transport equations in Cartesian coordinates for three domains (ice, water column, and sediments) in the ocean. 
          The dynamics include diffusion and advection part. 
          It is implemented in Fortran2003.}
        \resumeItem{Waves Groupiness in the Baltic See study}
          {The \href{https://doi.org/10.3103/S1068373916110054}{study} uses cluster analysis to classificate waves to groups and then uses Fourier and Wavelet analysis to study properties and features of these groups}
      \resumeItemListEnd

  \resumeSubHeadingListEnd


%-----------PUBLICATIONS--------------
\section{Recent publications}
  \resumeSubHeadingListStart
    \resumePubItem{}
      {Yakubov, S.; Protsenko, E. Alkalinity Generation in the Coastal Area, the Case of the Wadden Sea. Preprints 2021, 2021020036 (\href{https://doi.org/10.20944/preprints202102.0036.v1}{doi:10.20944/preprints202102.0036.v1})}
    \resumePubItem{}
      {Yakushev, E.V.; Wallhead, P.; Renaud, P.E.; Ilinskaya, A.; Protsenko, E.; Yakubov, S.; Pakhomova, S.; Sweetman, A.K.; Dunlop, K.; Berezina, A.; Bellerby, R.G.J.; Dale, T. Understanding the Biogeochemical Impacts of Fish Farms Using a Benthic-Pelagic Model. Water 2020, 12, 2384. (\href{https://doi.org/10.3390/w12092384}{doi:10.3390/w12092384})}
    \resumePubItem{}
      {Yakubov, S.; Wallhead, P.; Protsenko, E.; Yakushev, E.; Pakhomova, S.; Brix, H. A 1-Dimensional Sympagic–Pelagic–Benthic Transport Model (SPBM): Coupled Simulation of Ice, Water Column, and Sediment Biogeochemistry, Suitable for Arctic Applications. Water 2019, 11, 1582. (\href{https://doi.org/10.3390/w11081582}{doi:10.3390/w11081582})}
  \resumeSubHeadingListEnd


%-----------EDUCATION-----------------
\section{Education}
  \resumeSubHeadingListStart
    \resumeSubheading
      {Moscow State University}{Moscow, Russia}
      {Specialist, Oceanography}{2003 -- 2008}
  \resumeSubHeadingListEnd

%
%--------PROGRAMMING SKILLS------------
%\section{Programming Skills}
%  \resumeSubHeadingListStart
%    \item{
%      \textbf{Languages}{: Scala, Python, Javascript, C++, SQL, Java}
%      \hfill
%      \textbf{Technologies}{: AWS, Play, React, Kafka, GCE}
%    }
%  \resumeSubHeadingListEnd


%-------------------------------------------
\end{document}
